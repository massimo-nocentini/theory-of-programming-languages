% $Header:
% /home/vedranm/bitbucket/beamer/solutions/conference-talks/conference-ornate-20min.en.tex,v
% 90e850259b8b 2007/01/28 20:48:30 tantau $

\documentclass[8pt]{beamer}

% This file is a solution template for:

% - Talk at a conference/colloquium.
% - Talk length is about 20min.
% - Style is ornate.



% Copyright 2004 by Till Tantau <tantau@users.sourceforge.net>.
%
% In principle, this file can be redistributed and/or modified under
% the terms of the GNU Public License, version 2.
%
% However, this file is supposed to be a template to be modified
% for your own needs. For this reason, if you use this file as a
% template and not specifically distribute it as part of a another
% package/program, I grant the extra permission to freely copy and
% modify this file as you see fit and even to delete this copyright
% notice. 


\mode<presentation>
{
  \usetheme{Warsaw}
  % or ...

  \setbeamercovered{transparent}
  % or whatever (possibly just delete it)
}

\usepackage[english]{babel}
% or whatever

\usepackage[latin1]{inputenc}
% or whatever

\usepackage{times}
\usepackage[T1]{fontenc}
% Or whatever. Note that the encoding and the font should match. If T1
% does not look nice, try deleting the line with the fontenc.
% \lstset{
%   language=XML,
%   basicstyle=\footnotesize,%\ttfamily,
%   columns=fullflexible,
%   morekeywords={encoding,
%     xs:schema,xs:element,xs:complexType,xs:sequence,xs:attribute}
% }

\title[Analisi di reti metaboliche] {Analisi di reti metaboliche
  basata su propriet\`a di connessione}

% \subtitle
% {Include Only If Paper Has a Subtitle}

\author[Massimo Nocentini] % (optional, use only with lots of authors)
{Massimo~Nocentini\\\texttt{massimo.nocentini@gmail.com}}
% - Give the names in the same order as the appear in the paper.
% - Use the \inst{?} command only if the authors have different
%   affiliation.

 \institute[UniversitaStudiFirenze] % (optional, but mostly needed)
 { Universit\`a degli Studi di Firenze }
%   \inst{1}%
%   Department of Computer Science\\
%   University of Somewhere
%   \and
%   \inst{2}%
%   Department of Theoretical Philosophy\\
%   University of Elsewhere}
% - Use the \inst command only if there are several affiliations.
% - Keep it simple, no one is interested in your street address.

\date[CoursePresentation] % (optional, should be abbreviation of conference name)
{Firenze, \today}
% - Either use conference name or its abbreviation.
% - Not really informative to the audience, more for people (including
%   yourself) who are reading the slides online

\subject{Theory of Programming Languages}
% This is only inserted into the PDF information catalog. Can be left
% out. 



% If you have a file called "university-logo-filename.xxx", where xxx
% is a graphic format that can be processed by latex or pdflatex,
% resp., then you can add a logo as follows:

% \pgfdeclareimage[height=1.5cm]{university-logo}{logo/unifi}
% \logo{\pgfuseimage{university-logo}}

% Delete this, if you do not want the table of contents to pop up at
% the beginning of each subsection:
\AtBeginSubsection[]
{
  \begin{frame}<beamer>{Contenuti}
    \tableofcontents[currentsection,currentsubsection]
  \end{frame}
}


% If you wish to uncover everything in a step-wise fashion, uncomment
% the following command: 

%\beamerdefaultoverlayspecification{<+->}


\begin{document}

\begin{frame}[plain]
  \titlepage
  % \begin{center}
  %   \includegraphics[scale=.065]{logo/unifi}
  % \end{center}
\end{frame}

\begin{frame}{Contenuti}
  \tableofcontents[pausesections]
  % You might wish to add the option [pausesections]
\end{frame}


% Structuring a talk is a difficult task and the following structure
% may not be suitable. Here are some rules that apply for this
% solution: 

% - Exactly two or three sections (other than the summary).
% - At *most* three subsections per section.
% - Talk about 30s to 2min per frame. So there should be between about
%   15 and 30 frames, all told.

% - A conference audience is likely to know very little of what you
%   are going to talk about. So *simplify*!
% - In a 20min talk, getting the main ideas across is hard
%   enough. Leave out details, even if it means being less precise than
%   you think necessary.
% - If you omit details that are vital to the proof/implementation,
%   just say so once. Everybody will be happy with that.

\section{Motivazioni}

\subsection{Analisi di reti metaboliche}

\begin{frame}{Definizione di rete metabolica}
  % - A title should summarize the slide in an understandable fashion
  %   for anyone how does not follow everything on the slide itself.
  \begin{block}{Definizione}
    Una rete metabolica \`e un insieme di reazioni chimiche che si
    verificano all'interno di una cellula
  \end{block}
  \`E interessante studiarle per:
  \begin{itemize}
  \item capire le propriet\`a fisiologiche e biochimiche delle
    cellule
  \item ricostruire le reazioni che avvengono all'interno di
    organismi, sia batteri che esseri umani
  \item studiare il comportamento della cellula in relazione al
    contesto che la ospita
  \end{itemize}
\end{frame}

\begin{frame}{Codifica delle reti}%{usando il linguaggio \emph{SBML}}
  \textbf{SBML} (\textbf{S}ystems \textbf{B}iology \textbf{M}arkup
  \textbf{L}anguage) \`e un linguaggio per descrivere sistemi oggetto
  di trasformazioni fisiche
%   \begin{figure}
%     \includegraphics[scale=.4]{images/sbml-code-chunk.eps}
%   \end{figure}
\end{frame}

\subsection{Il problema delle storie}

\begin{frame}{Dalla rete al grafo}
  Data una reazione \emph{non reversibile} $r$ tale che $reagenti(r) =
  \{ r_{1}, \ldots, r_{n} \} \wedge prodotti(r) = \{ p_{1}, \ldots,
  p_{m} \}$, costruiamo il sotto grafo $reagenti(r) \times
  prodotti(r)$
    \begin{block}{Esempio}
      Se $reagenti(r) = \{ a, b, c, d \} \wedge prodotti(r) = \{a, e,
      f\}$ allora
%      \begin{figure}
%        \centering
%        \includegraphics[scale=.6]{images/non-reversible-reaction-example.dot.eps}
%        \label{fig:non-reversible-reaction-mapping}
%      \end{figure}
    \end{block}
\end{frame}

\begin{frame}{Seed Compounds}
  \begin{itemize}
  \item cercare le componenti fortemente connesse
  \item selezionare vertici in componenti sorgenti come \emph{seed
      compounds}
  \end{itemize}
%  \begin{figure}
%    \includegraphics[scale=.35]{images/biology-scc-decomposition.eps}
%  \end{figure}

\end{frame}

\begin{frame}{Storie}
  \begin{block}{Definizione}
    Dato un grafo orientato $G = (\mathbb{B} \cup \mathbb{W}, E)$, una
    \emph{storia} \`e un sotto grafo aciclico $G' = (\mathbb{B} \cup
    \mathbb{W'}, E')$ di $G$ tale che $E' \subseteq E $ e
  \begin{displaymath}
    \mathbb{W'} = \{w \in \mathbb{W}: indeg(w) > 0 \wedge outdeg(w)
    > 0\}
  \end{displaymath}
\end{block}
$\mathbb{B}$: vertici a cui \`e permesso essere sorgenti o pozzi
%\begin{center}
%  \includegraphics[scale=0.4]{images/graph-with-b-setted.eps}
%  \includegraphics[scale=0.4]{images/story-example.eps} 
%  \includegraphics[scale=0.4]{images/non-story-example.eps}
%\end{center} 
\end{frame}

\section{Nostro contributo}

\subsection{Obiettivi}

\begin{frame}{Analizzare reti, costruire $\mathbb{B}$ e verificare il
    metodo}
Gli obiettivi del nostro lavoro sono:
\begin{itemize}
\item<1-> rappresentare una rete mediante un grafo\\
  \footnotesize{astraendo dai molti dettagli di \emph{SBML}}
\item<2-> fornire strumenti per analizzare insiemi di reti\\
  \footnotesize{per avere informazioni sui metaboliti che appaiono in
    pi\`u di una rete}
\item<3-> costruire in modo automatico l'insieme $\mathbb{B}$\\
  \footnotesize{sfruttando le informazioni di tutte le reti studiate}
\item<4-> verificare se il metodo \`e accettabile\\
  \footnotesize{per misurare la validit\`a di $\mathbb{B}$}
\end{itemize}
\end{frame}

\subsection{Metodologia}

\begin{frame}{Rappresentazione della rete con grafo}
%\begin{center} 
%  \includegraphics[scale=.3]{images/sbml-code-chunk.eps}
%  \includegraphics[scale=.4]{images/presentation-input-graph-all-whites.eps}
%\end{center}
\end{frame}

\subsection{Risultati}

\begin{frame}{Analisi delle reti}
%  \begin{center}
%    \includegraphics[scale=.5]{images/ResultViewer-table-with-average-row-selected-particular.eps}
%  \end{center}
Tutte le reti hanno una struttura a ``clessidra'':
\begin{itemize}
\item molte componenti \emph{sorgenti} contenenti pochi vertici
\item poche componenti \emph{intermedie} contenenti molti vertici
\item molte componenti \emph{pozzo} contenenti pochi vertici
\end{itemize}
\pause
\begin{alertblock}{Troppi vertici sorgenti e pozzi}
  L'obiettivo sarebbe stato averne pochi
\end{alertblock}
\end{frame}

\begin{frame}{Analisi dei vertici}
%  \begin{center}
%    \includegraphics[scale=.5]{images/ResultViewer-grouping-table-zoom}
%  \end{center}
  Studiando un insieme di reti, pi\`u del 10\% dei vertici non hanno
  un ruolo univoco
  \pause
  \begin{alertblock}{Incertezza sulla costruzione di $\mathbb{B}$}
    Vertici con pi\`u di un ruolo inducono incertezza sul decidere la
    loro appartenenza all'insieme $\mathbb{B}$
  \end{alertblock}
\end{frame}

\begin{frame}{Combinazione delle due analisi}
%  \begin{center}
%      \includegraphics[scale=.5]{images/many-models-with-b-refinement}
%  \end{center}
  Se consideriamo tutti i vertici che hanno un ruolo univoco rispetto
  ad un insieme di reti, \`e possibile raffinare ogni $\mathbb{B}$
  escludendo i vertici che non hanno un ruolo conforme
\end{frame}


\section*{Sommario}

\begin{frame}{Riepilogo}
  % Keep the summary *very short*.
    \begin{itemize}
    \item costruire l'insieme $\mathbb{B}$ in modo automatico
      attraverso le componenti fortemente connesse \emph{non} sembra
      produrre risultati soddisfacenti
    \item l'unico caso utile riguarda l'analisi di modelli singoli,
      anche se in molti casi rimangono molti vertici in $\mathbb{B}$
    \end{itemize}
\end{frame}

\begin{frame}{Ricerca delle componenti fortemente connesse}
  Attualmente l'insieme $\mathbb{B}$ viene costruito in base a
  osservazioni e studi \emph{empirici}, costituito da metaboliti con
  propriet\`a particolari.

  Per costruirlo in modo automatico partizioniamo la rete in
  componenti connesse in quanto:
\begin{itemize}
\item<2-> \`e difficile assegnare il ruolo ad ogni vertice studiando
  l'intera rete date le sue dimensioni
\item<3-> \`e possibile astrarre dai cicli ed identificare classi di
  metaboliti equivalenti
\item<4-> due metaboliti equivalenti si producono a vicenda, pertanto
  gli associamo il ruolo della componente che li contiene
\item<5-> se una componente \`e \emph{sorgente} o \emph{pozzo} nel
  ``meta grafo'' allora aggiungiamo i vertici che la compongono in
  $\mathbb{B}$
\end{itemize}
\end{frame}

\end{document}

