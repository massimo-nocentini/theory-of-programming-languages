% $Header:
% /home/vedranm/bitbucket/beamer/solutions/conference-talks/conference-ornate-20min.en.tex,v
% 90e850259b8b 2007/01/28 20:48:30 tantau $

\documentclass[8pt]{beamer}

% This file is a solution template for:

% - Talk at a conference/colloquium.
% - Talk length is about 20min.
% - Style is ornate.



% Copyright 2004 by Till Tantau <tantau@users.sourceforge.net>.
%
% In principle, this file can be redistributed and/or modified under
% the terms of the GNU Public License, version 2.
%
% However, this file is supposed to be a template to be modified
% for your own needs. For this reason, if you use this file as a
% template and not specifically distribute it as part of a another
% package/program, I grant the extra permission to freely copy and
% modify this file as you see fit and even to delete this copyright
% notice. 


\mode<presentation>
{
  \usetheme{Warsaw}
  % or ...

  \setbeamercovered{transparent}
  % or whatever (possibly just delete it)
}

\usepackage[english]{babel}
% or whatever

\usepackage[latin1]{inputenc}
% or whatever
\usepackage{wasysym}
\usepackage{times}
\usepackage[T1]{fontenc}
\usepackage{amsmath}
\usepackage{cancel}
% Or whatever. Note that the encoding and the font should match. If T1
% does not look nice, try deleting the line with the fontenc.
% \lstset{
%   language=XML,
%   basicstyle=\footnotesize,%\ttfamily,
%   columns=fullflexible,
%   morekeywords={encoding,
%     xs:schema,xs:element,xs:complexType,xs:sequence,xs:attribute}
% }
\setbeamersize{text margin left=.5cm,text margin right=.5cm} 
\title[Type reconstruction] {Type reconstruction}

\subtitle{constraint-based, unification and a little interpreter}

\author[Massimo Nocentini] % (optional, use only with lots of authors)
{Massimo~Nocentini\\\texttt{massimo.nocentini@gmail.com}\\~\\~\\
    Prof. Battistina Venneri}
% - Give the names in the same order as the appear in the paper.
% - Use the \inst{?} command only if the authors have different
%   affiliation.
 
 \institute[UniversitaStudiFirenze] % (optional, but mostly needed)
 { Universit\`a degli Studi di Firenze }
%   \inst{1}%
%   Department of Computer Science\\
%   University of Somewhere
%   \and
%   \inst{2}%
%   Department of Theoretical Philosophy\\
%   University of Elsewhere}
% - Use the \inst command only if there are several affiliations.
% - Keep it simple, no one is interested in your street address.

\date[CoursePresentation] % (optional, should be abbreviation of conference name)
{Firenze, \today}
% - Either use conference name or its abbreviation.
% - Not really informative to the audience, more for people (including
%   yourself) who are reading the slides online

\subject{Theory of Programming Languages}
% This is only inserted into the PDF information catalog. Can be left
% out. 



% If you have a file called "university-logo-filename.xxx", where xxx
% is a graphic format that can be processed by latex or pdflatex,
% resp., then you can add a logo as follows:

% \pgfdeclareimage[height=1.5cm]{university-logo}{logo/unifi}
% \logo{\pgfuseimage{university-logo}}

% Delete this, if you do not want the table of contents to pop up at
% the beginning of each subsection:
\AtBeginSubsection[]
{
  \begin{frame}<beamer>{Contents}
    \tableofcontents[currentsection,currentsubsection]
  \end{frame}
}


% If you wish to uncover everything in a step-wise fashion, uncomment
% the following command: 

%\beamerdefaultoverlayspecification{<+->}


\begin{document}

\begin{frame}[plain]
  \titlepage
   \begin{center}
     \includegraphics[scale=.065]{logo/unifi}
   \end{center}
\end{frame}

\begin{frame}{Road map}
  \tableofcontents[pausesections]
  % You might wish to add the option [pausesections]
\end{frame}


% Structuring a talk is a difficult task and the following structure
% may not be suitable. Here are some rules that apply for this
% solution: 

% - Exactly two or three sections (other than the summary).
% - At *most* three subsections per section.
% - Talk about 30s to 2min per frame. So there should be between about
%   15 and 30 frames, all told.

% - A conference audience is likely to know very little of what you
%   are going to talk about. So *simplify*!
% - In a 20min talk, getting the main ideas across is hard
%   enough. Leave out details, even if it means being less precise than
%   you think necessary.
% - If you omit details that are vital to the proof/implementation,
%   just say so once. Everybody will be happy with that.

\begin{frame}{Syntax review}
    \begin{block}{Terms}
        \begin{center}
            \begin{tabular}{ r l r }
              $t ::= $ & $x$ & \emph{variable} \\
               & $\lambda x:T.t$ & \emph{abstraction} \\
               & $t \, t$ & \emph{application} \\
            \end{tabular}
        \end{center}
    \end{block}
    \begin{block}{Values}
        \begin{center}
            \begin{tabular}{ r l r }
              $v ::= $ & $\lambda x:T.t$ & \emph{abstraction value} \\
            \end{tabular}
        \end{center}
    \end{block}
    \begin{block}{Types}
        \begin{center}
            \begin{tabular}{ r l r }
              $T ::= $ & $T \rightarrow T$ & \emph{function type} \\
            \end{tabular}
        \end{center}
    \end{block}
    \begin{block}{Contexts}
        \begin{center}
            \begin{tabular}{ r l r }
              $\Gamma ::= $ & $\emptyset$ & \emph{empty context} \\
               & $\Gamma, x:T$ & \emph{variable binding} \\
            \end{tabular}
        \end{center}
    \end{block}
\end{frame}

\begin{frame}{Evaluation rules review}
    \begin{block}{Evaluation rules}

        \begin{center}
            \begin{tabular}{ c r }
                    $\displaystyle {{t_{1} \rightarrow t_{1}^{\prime}}\over
                    {t_{1} \, t_{2} \rightarrow t_{1}^{\prime} \, t_{2}}}
                $ & (\emph{E-APP1}) \\ 
               & \\
               $\displaystyle{{t_{2} \rightarrow t_{2}^{\prime}}\over
                    {v_{1} \, t_{2} \rightarrow v_{1} \, t_{2}^{\prime}}}$ &
                         (\emph{E-APP2}) \\
               & \\ 
               $\displaystyle(\lambda x:T_{11}.t_{12})v_{2} \rightarrow 
                [x \mapsto v_{2}]t_{12}$ & (\emph{E-APPABS}) \\
            \end{tabular}
        \end{center}
    \end{block}
\end{frame}

\begin{frame}{Typing rules review and Inversion lemma}
    \begin{block}{Typing rules}
        \begin{center}
            \begin{tabular}{ c r }
                    $\displaystyle {{x:T \in \Gamma}\over
                    {\Gamma \vdash x:T}} $ & (\emph{T-VAR}) \\ 
               & \\
               $\displaystyle{{\Gamma, x:T_{1} \vdash t_{2}:T_{2}}\over
                    {\Gamma \vdash \lambda x:T_{1}.t_{2} : T_{1} 
                        \rightarrow T_{2}}} $ & (\emph{T-ABS}) \\
               & \\ 
               $\displaystyle{{\Gamma \vdash t_{1}:T_{11} \rightarrow T_{12} \quad
                     \Gamma \vdash t_{2}:T_{11}}\over
                    {\Gamma \vdash t_{1} \, t_{2} : T_{12}}} $ & (\emph{T-APP}) \\
            \end{tabular}
        \end{center}
    \end{block}
    \begin{block}{Inversion Lemma}
        \begin{itemize}
            \item if $\Gamma \vdash x:S_1$ then $x:S_1\in\Gamma$
            \item if $\Gamma \vdash \lambda x:S_1.t_2:S_3$ then $S_3 =
            S_1\rightarrow S_2$ for some $S_2$ such that $\Gamma,
            x:S_1\vdash t_2:S_2$
            \item if $\Gamma \vdash t_1 \, t_2:S_2$ then there exists $S_1$
            such that $\Gamma \vdash t_1: S_1\rightarrow S_2$ and
            $\Gamma \vdash t_2:S_1$
        \end{itemize}
    \end{block}
\end{frame}

\section{Introduction}

\subsection{Type variables and substitutions}

\begin{frame}{Introduction}
    Up to now we've worked with terms which all depends on
    \emph{explicit} type annotation, that is every $\lambda$-
    abstraction have to declare the \emph{concrete} type for the variable
    it introduce. \\~\\
    
    It should be interesting to be ``lazy'' (in the fisical sense, not
    because we like to be some instance of a \emph{lazy} structures 
    \smiley) and say:
    \begin{center}
        ``I don't care to declare a \emph{concrete} type for my variable now,\\
        suspend the decision and call this type $X$, I'll specify it later''
    \end{center}
     
    Suppose we're deeply ``lazy'', why not apply the previous though
    to \emph{all} variables we introduce in our program? When are our terms 
    well typed if no concrete type is declared at all? \\~\\
    
    In this lecture we'll see a method which allow us to be ``deeply lazy''
    while been able to type our meaningful terms and discard divergent ones
    (\texttt{lambda x.(x x)}, remember?)
\end{frame}

\begin{frame}{Augment type set with type variables}
    In order to leave unspecified the type for a variable we've to augment
    our language with a new type category, which we'll call \emph{type
    variables}, written as $\mathcal{A}$:
    \begin{block}{Types - augmented with type variables}
        \begin{center}
            \begin{tabular}{ r l l }
              $T ::= $ & $T \rightarrow T$ & \emph{function type} \\
               & $\mathcal{A}$ & \emph{type variable} \\
              $\mathcal{A} ::= $ & $\{A,B,\ldots,X_{i \in \mathbb{N}},
                \ldots \}$ & \emph{type variables} \\
            \end{tabular}
        \end{center}
    \end{block}
    
    Let $X \in \mathcal{A}$:
    \begin{itemize}
        \item there's no typing rule that uses the category $\mathcal{A}$
        \item $X$ can represent a \emph{basic} type
            (ie. \texttt{Bool}, \texttt{Nat}, \ldots) or another 
            unspecified type
        \item $X$, being a type, can be used by the defined typing rules
        \item $X \not = Y, \forall Y \in \mathcal{A} \setminus \{X\}$, 
            in other words, $\mathcal{A}$ is infinite and types represented
            by different type variables are different too
    \end{itemize}
    
    \begin{example}
        \begin{displaymath}
            \begin{split}
                \Gamma &\vdash \lambda x:X.x : X \rightarrow X\\
                \Gamma &\vdash \lambda x:A.x : A \rightarrow A\\
                \Gamma &\vdash \lambda s:S. \lambda z: Z. (s\, (s\, z)) : 
                    (Z \rightarrow Z) \rightarrow Z \rightarrow Z\\
            \end{split}
        \end{displaymath}
    \end{example}
\end{frame}

\begin{frame}{Substitutions}
    \begin{center}
        ``I'd like that type variable $X$ in my term $t$ to stands for 
        \texttt{Nat} type. \\Is it possible to do a \emph{substitution}?''
    \end{center}

    
    \begin{block}{Definition of substitution}
        A \emph{type substitution} $\sigma$ is a mapping $\sigma: \mathcal{A}
        \rightarrow T$ \\~\\
        A \emph{substitution application} $\sigma S_{1}$ 
        of a type substitution $\sigma$ to a type $S_{1}$ is defined 
        inductively on the structure of $S_1$:
        \begin{displaymath}
            \begin{split}
                X \in \mathcal{A} \wedge \exists T_1: (X \mapsto T_{1}) 
                    \in \sigma & \rightarrow \sigma X = T_{1} \\ 
                X \in \mathcal{A} \wedge \forall T_1: (X \mapsto T_{1}) 
                    \not \in \sigma & \rightarrow \sigma X = X \\ 
                T_{1} \text{ is a concrete type } &\rightarrow  
                    \sigma T_{1} = T_{1}\\
                T_{1}, T_{2} \in T & \rightarrow \sigma(T_{1} \rightarrow 
                    T_{2}) = \sigma T_{1} \rightarrow \sigma T_{2} \\ 
            \end{split}
        \end{displaymath}
    \end{block}
\end{frame}

\begin{frame}{Typing relation is closed under substitution application}
    For the following it is useful to introduce two combinations 
    (let $\mathcal{C}$ is a set of contexts):
    \begin{itemize}
        \item $\Gamma = \{ x_1:T_1, \ldots, x_n:T_n\} \in \mathcal{C} 
            \rightarrow \sigma\Gamma = (x_1:\sigma T_1, \ldots,
                x_n:\sigma T_n) \in \mathcal{C}$
        \item let $\sigma, \gamma$ be two type substitutions, their
            composition $\sigma \circ \gamma$ is defined as follows:
            \begin{displaymath}
                \begin{split}
                    \sigma \circ \rho &= \{X \mapsto \sigma T_1 | \,
                        \forall T_1:(X \mapsto T_1) \in \rho \} \cup \\
                    &\quad \{X \mapsto T_1 |\, 
                        \forall T_{1}:\left( (X \mapsto T_1) \in \sigma \wedge
                            (X \mapsto T_1) \not \in \rho\right)\} \\
                \end{split}
            \end{displaymath}
            
    \end{itemize}
    \begin{theorem}
        Let $\sigma$ be a type substitution, $\Gamma$ a context, 
        $T_1 \in T$ and $t$ a term. Then:
        \begin{displaymath}
            \Gamma \vdash t:T_1 \rightarrow \sigma\Gamma \vdash \sigma t: \sigma T_1
        \end{displaymath}
    \end{theorem}
\end{frame}

\subsection{Parametric polymorphism and type inference}

\begin{frame}{Parametric Polymorphism}
    Let $\Gamma$ be a context and $t$ be a term containing free variables.
    A first question arises:
    \begin{itemize}
        \item $\forall \sigma. \exists T_1: 
            \sigma \Gamma \vdash \sigma t : T_1$\\
            {\footnotesize Do all type sustitution $\sigma$ makes the term 
                $\sigma t$ typeable under assumption $\sigma \Gamma$?}\\~\\
            If this is the case, why not write \emph{any} concrete type
            using a type variable and use a substitution to obtain the 
            original one? 
            
            \begin{example}
                Write $\lambda s:Z \rightarrow Z.\lambda z:Z.(s\, (s\, z))$
                instead of $\lambda s:Nat \rightarrow Nat.
                    \lambda z:Nat.(s\, (s\, z))$ \\~\\
                For example, if we work with concrete type $String$ just use 
                $\sigma = \{Z \mapsto String\} $
            \end{example}
            
            Holding type variables abstract during type checking is called
            \emph{parametric polymorphism}: type variables are used to allow
            a term in which they appear usable in many concrete contexts
    \end{itemize}
\end{frame}

\begin{frame}{Type inference}
    Let $\Gamma$ be a context and $t$ be a term containing free variables.
    A second question arises:
    \begin{itemize}
        \item $\exists \sigma. \exists T_1: 
            \sigma \Gamma \vdash \sigma t : T_1$\\
            {\footnotesize Is it always possible to find a type 
                sustitution $\sigma$ such that the term 
                $\sigma t$ is typeable under assumption $\sigma \Gamma$?}\\~\\
            If this is the case, suppose $\not \exists T_{1}.
            \Gamma \vdash t:T_1$, using 
            type substitution $\sigma$ we're able to give a type $T_{2}$ 
            to $\sigma t$, formally $\sigma \Gamma \vdash \sigma t: T_{2}$
            
            \begin{example}
                $\lambda s:S.\lambda z:Z.(s\, (s\, z))$ 
                has {\color{red} no} type, $\forall S,Z \in \mathcal{A}$\\~\\
                but it has type if we use $\sigma = \{
                    S \mapsto Nat \rightarrow Nat, Z \mapsto Nat\}$ or
                    $\sigma = \{S \mapsto Z \rightarrow Z\}$
            \end{example}
            
            Looking for valid ``instantiations'' of type variables is called
            \emph{type inference}: the compiler fill in type information 
            wherever the user don't specify them
    \end{itemize}
\end{frame}

\subsection{Definition of solution for $(\Gamma, t)$}

\begin{frame}
    Let $\Gamma$ be a context and $t$ be a term containing free variables.
    \begin{block}{Definition of \emph{solution} for $(\Gamma, t)$}
        Let $\sigma$ be a type substitution and $T_{1} \in T$.\\
        A  \emph{solution} for $(\Gamma, t)$ is a pair $(\sigma, T_{1})$
        such that $\sigma \Gamma \vdash \sigma t:T_{1}$
    \end{block}
    
    \begin{example}
        Let $\Gamma = \{f:X, a:Y\}$ and $t = f\, a$ then:
        \begin{displaymath}
            ([X \mapsto Nat \rightarrow Nat, Y \mapsto Nat], Nat) \quad 
            ([X \mapsto Y \rightarrow Z], Z)
        \end{displaymath}
        are both solutions for $(\Gamma, t)$.
    \end{example}
\end{frame}

\section{Constrait-based typing}
\subsection{Constraint set and relations}
\begin{frame}
    Questions:
    \begin{itemize}
        \item During an execution of a type-checking algorithm, why
                    not record constraints of the form $S_i = T_i$ instead 
                    to actually perform a type comparison?
        \item As we've seen in the last example, there exists countable 
                    solutions for a pair $(\Gamma, t)$, are they related in 
                    some way? 
    \end{itemize}
    
    \begin{block}{Definition of \emph{constraint set} 
                    and \emph{unify} relation}
        A \emph{constraint set} $\mathcal{C}$ is a set of equations 
        $\{S_i \triangleq T_i \}_{i \in \mathbb{N}}$ such that 
        $\mathcal{C} \subseteq T \times T$
        \\~\\
        A type substitution $\sigma$ \emph{unify} an equation 
        $S \triangleq T$, written as $\sigma \Join S \triangleq T$, 
        if $\sigma S = \sigma T$
        \\~\\
        Extending the unification relation to constraint sets, a type substitution
        $\sigma$ \emph{unify} a constraint set $\mathcal{C}$, written as 
        $\sigma \Join \mathcal{C}$, if $\forall (S \triangleq T)
        \in \mathcal{C}: \sigma \Join (S \triangleq T)$ 
    \end{block}
    
    Answers:
    \begin{itemize}
        \item The modified type checking algorithm prove that a 
            term $t$ has type $T_1$ under assumptions $\Gamma$ 
            whenever there exists a type substitution $\sigma$ such that 
            $\sigma \Join \mathcal{C}$
        \item let $\mathcal{C}$ be a constraint set for a pair $(\Gamma, t)$.
            Two solutions $(\sigma_1,T_1)$ and $(\sigma_2,T_2)$ are related if  
            $\sigma_1 \Join \mathcal{C}$ and $\sigma_2 \Join \mathcal{C}$ 
    \end{itemize}
\end{frame}


\begin{frame}
    Let's see an example of how we collect a constraint $S \triangleq T$
    \begin{example}
        Suppose to have an application term $t_1 t_2$ with $\Gamma \vdash t_1 : T_1$
        and $\Gamma \vdash t_2 : T_2$ \\~\\
        Instead of checking:
        \begin{itemize}
            \item $T_1$ has the form $T_2 \rightarrow T_{12}$, for some $T_{12}$
            \item $t_1 t_2$ has type \emph{exactly} $T_{12}$
        \end{itemize}
        We suspend a decision for the type $T_{12}$, abstracting it with a 
        \emph{fresh} type variable $X$, creating the constraint 
        $T_1 \triangleq T_2 \rightarrow X$ (hence $t_1 t_2$
        has type $X$ from now on!)
    \end{example}
    
    The following questions drive what follow:
    \begin{itemize}
        \item given a pair $(\Gamma, t)$ there always exist a constraint set?
        \item suppose a constraint set exists, is it unique?
        \item assume the existence is enough, how is its construction defined
                for all cases?
        \item how can we use the constraint set to build a solution for $(\Gamma, t)$?
    \end{itemize}
\end{frame}

\subsection{Constraint typing relation}

\begin{frame}
    \begin{block}{Definition of \emph{constraint typing relation} 
            $\Gamma \vdash t : T_1 \; \textbar_\mathcal{X} \; \mathcal{C}$}
        The \emph{constraint typing relation} where a term $t$ has type
            $T_1$ under assumptions $\Gamma$ 
            whenever exists a type substitution $\sigma$ such that 
            $\sigma \Join \mathcal{C}$, written as
            $\Gamma \vdash t : T_1 \textbar_\mathcal{X} \mathcal{C}$,
            is defined inductively as follow:
            \begin{center}
                \begin{tabular}{ c r }
                        $\displaystyle {{x:T \in \Gamma}\over
                        {\Gamma \vdash x:T \; \textbar_{\emptyset}\;  \emptyset}} $ 
                            & (\emph{CT-VAR}) \\ 
                   & \\
                   $\displaystyle{{\Gamma, x:T_{1} \vdash t_{2}:T_{2}\;
                    \textbar_{\mathcal{X}} \; \mathcal{C} }\over
                        {\Gamma \vdash \lambda x:T_{1}.t_{2} : T_{1} 
                            \rightarrow T_{2}\;
                    \textbar_{\mathcal{X}} \; \mathcal{C} }} $ & (\emph{CT-ABS}) \\
                   & \\ 
                   ${\displaystyle{{{
                        \begin{array}{rl}
                         \text{let} & X \text{ be fresh variable} \\
                        \Gamma \vdash & t_{1}:T_{1} \; \textbar_{\mathcal{X}_1} \;
                            \mathcal{C}_1  \\
                        \Gamma \vdash & t_{2}:T_{2} \; \textbar_{\mathcal{X}_2} \;
                            \mathcal{C}_2  \\
                         \mathcal{C}^{\prime} = & \mathcal{C}_1 \cup \mathcal{C}_2
                            \cup \{ T_1 \triangleq T_2 \rightarrow X \}
                        \end{array}}\over
                        {\Gamma \vdash t_{1} \, t_{2} : X\;
                    \textbar_{\mathcal{X}_1 \cup \mathcal{X}_2 \cup
                        \{X\}} \; \mathcal{C}^{\prime} }}}} $ & (\emph{CT-APP}) \\
                \end{tabular}
            \end{center}
        The set $\mathcal{X}$ is used to track type variables introduced
            by applications of rule \emph{CT-APP}
    \end{block}
\end{frame}

\begin{frame}{Extended example}
\footnotesize
    \begin{center}
        \begin{overprint}
            \onslide<1>
            \begin{tabular}{c}
                {}\\
                {}\\
                $\displaystyle {{}\over{\Gamma \vdash \lambda x:X.  
                \lambda y:Y.\lambda z:Z.  ((x\, z)\, (y \, 
                z)):S_1\,|_{\mathcal{X}}\, \mathcal{C}}}$
            \end{tabular}
            \onslide<2>
                \begin{tabular}{c}
                {}\\
                $\displaystyle {{{}\over{\Gamma, x:X \vdash \lambda 
                    y:Y.\lambda z:Z.  ((x\, z)\, (y \, 
                    z)):S_2\,|_{\mathcal{X}}\, \mathcal{C}}}} $\\
                $\displaystyle {{}
                    \over{\Gamma \vdash \lambda x:X. \lambda y:Y.\lambda z:Z.
                    ((x\, z)\, (y \, z)):X \rightarrow S_2
                \,|_{\mathcal{X}}\, \mathcal{C}}}$
                \end{tabular}
            \onslide<3>
                \begin{tabular}{c}
                $\displaystyle {{{}\over{\Gamma, x:X, y:Y \vdash 
                    \lambda z:Z.  ((x\, z)\, (y \, 
                    z)):S_3\,|_{\mathcal{X}}\, \mathcal{C}}}} $\\
                $\displaystyle {{{}\over{\Gamma, x:X \vdash \lambda 
                    y:Y.\lambda z:Z.  ((x\, z)\, (y \, z)):Y 
                    \rightarrow S_3\,|_{\mathcal{X}}\, \mathcal{C}}}} 
                    $\\
                $\displaystyle {{}
                    \over{\Gamma \vdash \lambda x:X. \lambda y:Y.\lambda z:Z.
                    ((x\, z)\, (y \, z)):X \rightarrow Y \rightarrow 
                S_3\,|_{\mathcal{X}}\, \mathcal{C}}}$
                \end{tabular}
            \onslide<4>
                \begin{tabular}{c}
                    $\displaystyle {{{}\over{\Gamma, x:X, y:Y, z:Z 
                    \vdash
                    (x\, z)\, (y \, z):S_4\,|_{\mathcal{X}}\, 
                    \mathcal{C}}}} $\\
                    $\displaystyle {{{}\over{\Gamma, x:X, y:Y \vdash 
                        \lambda z:Z.  ((x\, z)\, (y \, 
                        z)):Z\rightarrow S_4\,|_{\mathcal{X}}\, 
                        \mathcal{C}}}} $\\
                    $\displaystyle {{{}\over{\Gamma, x:X \vdash \lambda 
                        y:Y.\lambda z:Z.  ((x\, z)\, (y \, z)):Y 
                        \rightarrow Z \rightarrow 
                        S_4\,|_{\mathcal{X}}\, \mathcal{C}}}} $\\
                    $\displaystyle {{}
                        \over{\Gamma \vdash \lambda x:X. \lambda y:Y.\lambda z:Z.
                        ((x\, z)\, (y \, z)):X \rightarrow Y 
                    \rightarrow Z\rightarrow S_4\,|_{\mathcal{X}}\, 
                    \mathcal{C}}}$
                \end{tabular}
            \onslide<5>
                \begin{tabular}{c}
                    \begin{tabular}{cc}
                        \begin{tabular}{c}
                            $\displaystyle {{{}\over{\Gamma, x:X, z:Z 
                            \vdash
                            x\, z:S_1\,|_{\mathcal{X}_1}\, 
                            \mathcal{C}_1}}} $                        
                            \end{tabular}&
                        \begin{tabular}{c}
                            $\displaystyle {{{}\over{\Gamma, y:Y, z:Z 
                            \vdash
                            y \, z:S_2\,|_{\mathcal{X}_2}\, 
                            \mathcal{C}_2}}} $                        
                            \end{tabular}
                    \end{tabular}\\
                $\displaystyle {{{}\over{\Gamma, x:X, y:Y, z:Z \vdash
                (x\, z)\, (y \, 
                z):A\,|_{\mathcal{X}_1\cup\mathcal{X}_2\cup\{A\}}\, 
                \mathcal{C}_{1}\cup\mathcal{C}_{2}\cup\{S_1\triangleq S_2\rightarrow 
                A\}}}} $\\
                $\displaystyle {{{}\over{\Gamma, x:X, y:Y \vdash 
                    \lambda z:Z.  ((x\, z)\, (y \, z)):Z \rightarrow A 
                    \,|_{\mathcal{X}_1\cup\mathcal{X}_2\cup\{A\}}\, 
                    \mathcal{C}_{1}\cup\mathcal{C}_{2}\cup\{S_1\triangleq S_2\rightarrow 
                    A\}}}} $\\
                $\displaystyle {{{}\over{\Gamma, x:X \vdash \lambda 
                    y:Y.\lambda z:Z.  ((x\, z)\, (y \, z)):Y 
                    \rightarrow Z\rightarrow A\,|_{\mathcal{X}_1
                    \cup\mathcal{X}_2\cup\{A\}}\, 
                \mathcal{C}_{1}\cup\mathcal{C}_{2}\cup\{S_1\triangleq S_2\rightarrow 
                A\}}}}  $\\
                $\displaystyle {{}
                    \over{\Gamma \vdash \lambda x:X. \lambda y:Y.\lambda z:Z.
                    ((x\, z)\, (y \, z)):X \rightarrow Y \rightarrow 
                Z\rightarrow 
                A\,|_{\mathcal{X}_1\cup\mathcal{X}_2\cup\{A\}}\, 
                \mathcal{C}_{1}\cup\mathcal{C}_{2}\cup\{S_1\triangleq S_2\rightarrow 
                A\}}} $
                \end{tabular}
            \onslide<6>
                \begin{tabular}{c}
                    \begin{tabular}{cc}
                        \begin{tabular}{c}
                            \begin{tabular}{cc}
                                \begin{tabular}{c}
                                    $\displaystyle {{{}\over{\Gamma, 
                                    x:X \vdash x :S_3\,|_{\mathcal{X}_3}\, 
                                    \mathcal{C}_3}}} $
                                \end{tabular} &
                                \begin{tabular}{c}
                                    $\displaystyle {{{}\over{\Gamma, 
                                    z:Z \vdash z 
                                    :S_4\,|_{\mathcal{X}_4}\, 
                                    \mathcal{C}_4}}} $
                                \end{tabular}
                            \end{tabular}\\
                            $\displaystyle {{{}\over{\Gamma, x:X, z:Z 
                            \vdash
                            x\, 
                            z:B\,|_{\mathcal{X}_3\cup\mathcal{X}_4
                            \cup{\{B\}}}\, 
                            \mathcal{C}_3\cup\mathcal{C}_4\cup\{S_3
                            \triangleq  
                            S_4 \rightarrow B\}}}} $
                            \end{tabular}&
                            %put in the following tabular the previous 
                            %changes about nested tabular
                        \begin{tabular}{c}
                            $\displaystyle {{{}\over{\Gamma, y:Y, z:Z 
                            \vdash
                            y \, z:S_2\,|_{\mathcal{X}_2}\, 
                            \mathcal{C}_2}}} $                        
                            \end{tabular}
                    \end{tabular}\\
                $\displaystyle {{{}\over{\Gamma, x:X, y:Y, z:Z \vdash
                (x\, z)\, (y \, 
                z):A\,|_{\mathcal{X}_3\cup\mathcal{X}_4\cup\mathcal{X}_2
                    \cup\{A,B\}}\, 
                \mathcal{C}_{3}\cup\mathcal{C}_{4}\cup\mathcal{C}_{2}\cup\{B\triangleq S_2\rightarrow 
                A, S_3 \triangleq  S_4 \rightarrow B\}}}} $\\
                $\displaystyle {{{}\over{\Gamma, x:X, y:Y \vdash 
                    \lambda z:Z.  ((x\, z)\, (y \, z)):Z \rightarrow A 
                    \,|_{\mathcal{X}_3\cup\mathcal{X}_4\cup\mathcal{X}_2
                        \cup\{A,B\}}\, 
                    \mathcal{C}_{3}\cup\mathcal{C}_{4}\cup\mathcal{C}_{2}\cup\{B\triangleq S_2\rightarrow 
                    A, S_3 \triangleq  S_4 \rightarrow B\}}}} $\\
                $\displaystyle {{{}\over{\Gamma, x:X \vdash \lambda 
                    y:Y.\lambda z:Z.  ((x\, z)\, (y \, z)):Y 
                    \rightarrow Z\rightarrow 
                    A\,|_{\mathcal{X}_3\cup\mathcal{X}_4
                    \cup\mathcal{X}_2\cup\{A,B\}}\, 
                \mathcal{C}_{3}\cup\mathcal{C}_{4}\cup\mathcal{C}_{2}\cup\{B\triangleq S_2\rightarrow 
                A, S_3 \triangleq  S_4 \rightarrow B\}}}}  $\\
                $\displaystyle {{}
                    \over{\Gamma \vdash \lambda x:X. \lambda y:Y.\lambda z:Z.
                    ((x\, z)\, (y \, z)):X \rightarrow Y \rightarrow 
                Z\rightarrow 
                A\,|_{\mathcal{X}_3\cup\mathcal{X}_4\cup\mathcal{X}_2
                    \cup\{A,B\}}\, 
                \mathcal{C}_{3}\cup\mathcal{C}_{4}\cup\mathcal{C}_{2}\cup\{B\triangleq S_2\rightarrow 
                A, S_3 \triangleq  S_4 \rightarrow B\}}} $
                \end{tabular}
            \onslide<7>
                \begin{tabular}{c}
                    \begin{tabular}{cc}
                        \begin{tabular}{c}
                            \begin{tabular}{cc}
                                \begin{tabular}{c}
                                    $\displaystyle {{{}\over{x:X \in 
                                    \Gamma, x:X}}}$\\
                                    $\displaystyle {{{}\over{\Gamma, 
                                    x:X \vdash x :X\,|_{\emptyset}\, 
                                    \emptyset}}} $
                                \end{tabular} &
                                \begin{tabular}{c}
                                    $\displaystyle {{{}\over{\Gamma, 
                                    z:Z \vdash z 
                                    :S_4\,|_{\mathcal{X}_4}\, 
                                    \mathcal{C}_4}}} $
                                \end{tabular}
                            \end{tabular}\\
                            $\displaystyle {{{}\over{\Gamma, x:X, z:Z 
                            \vdash
                            x\, z:B\,|_{\mathcal{X}_4
                            \cup{\{B\}}}\, \mathcal{C}_4\cup\{X
                            \triangleq  S_4 
                            \rightarrow B\}}}} $
                            \end{tabular}&
                            %put in the following tabular the previous 
                            %changes about nested tabular
                        \begin{tabular}{c}
                            $\displaystyle {{{}\over{\Gamma, y:Y, z:Z 
                            \vdash
                            y \, z:S_2\,|_{\mathcal{X}_2}\, 
                            \mathcal{C}_2}}} $                        
                            \end{tabular}
                    \end{tabular}\\
                $\displaystyle {{{}\over{\Gamma, x:X, y:Y, z:Z \vdash
                (x\, z)\, (y \, 
                z):A\,|_{\mathcal{X}_4\cup\mathcal{X}_2
                    \cup\{A,B\}}\, 
                \mathcal{C}_{4}\cup\mathcal{C}_{2}\cup\{B\triangleq S_2\rightarrow 
                A, X \triangleq  S_4 \rightarrow B\}}}} $\\
                $\displaystyle {{{}\over{\Gamma, x:X, y:Y \vdash 
                    \lambda z:Z.  ((x\, z)\, (y \, z)):Z \rightarrow A 
                    \,|_{\mathcal{X}_4\cup\mathcal{X}_2
                        \cup\{A,B\}}\, 
                    \mathcal{C}_{4}\cup\mathcal{C}_{2}\cup\{B\triangleq S_2\rightarrow 
                    A, X \triangleq  S_4 \rightarrow B\}}}} $\\
                $\displaystyle {{{}\over{\Gamma, x:X \vdash \lambda 
                    y:Y.\lambda z:Z.  ((x\, z)\, (y \, z)):Y 
                    \rightarrow Z\rightarrow A\,|_{\mathcal{X}_4
                    \cup\mathcal{X}_2\cup\{A,B\}}\, 
                \mathcal{C}_{4}\cup\mathcal{C}_{2}\cup\{B\triangleq S_2\rightarrow 
                A, X \triangleq  S_4 \rightarrow B\}}}}  $\\
                $\displaystyle {{}
                    \over{\Gamma \vdash \lambda x:X. \lambda y:Y.\lambda z:Z.
                    ((x\, z)\, (y \, z)):X \rightarrow Y \rightarrow 
                Z\rightarrow A\,|_{\mathcal{X}_4\cup\mathcal{X}_2
                    \cup\{A,B\}}\, 
                \mathcal{C}_{4}\cup\mathcal{C}_{2}\cup\{B\triangleq S_2\rightarrow 
                A, X \triangleq  S_4 \rightarrow B\}}} $
                \end{tabular}
            \onslide<8>
                \begin{tabular}{c}
                    \begin{tabular}{cc}
                        \begin{tabular}{c}
                            \begin{tabular}{cc}
                                \begin{tabular}{c}
                                    $\displaystyle {{{}\over{x:X \in 
                                    \Gamma, x:X}}}$\\
                                    $\displaystyle {{{}\over{\Gamma, 
                                    x:X \vdash x :X\,|_{\emptyset}\, 
                                    \emptyset}}} $
                                \end{tabular} &
                                \begin{tabular}{c}
                                    $\displaystyle {{{}\over{z:Z \in 
                                    \Gamma, z:Z}}}$\\
                                    $\displaystyle {{{}\over{\Gamma, 
                                    z:Z \vdash z :Z\,|_{\emptyset}\, 
                                    \emptyset}}} $
                                \end{tabular}
                            \end{tabular}\\
                            $\displaystyle {{{}\over{\Gamma, x:X, z:Z 
                            \vdash
                            x\, z:B\,|_{{\{B\}}}\, \{X \triangleq  Z \rightarrow 
                            B\}}}} $
                            \end{tabular}&
                            %put in the following tabular the previous 
                            %changes about nested tabular
                        \begin{tabular}{c}
                            $\displaystyle {{{}\over{\Gamma, y:Y, z:Z 
                            \vdash
                            y \, z:S_2\,|_{\mathcal{X}_2}\, 
                            \mathcal{C}_2}}} $                        
                            \end{tabular}
                    \end{tabular}\\
                $\displaystyle {{{}\over{\Gamma, x:X, y:Y, z:Z \vdash
                (x\, z)\, (y \, z):A\,|_{\mathcal{X}_2
                    \cup\{A,B\}}\, 
                \mathcal{C}_{2}\cup\{B\triangleq S_2\rightarrow A, X
                \triangleq  Z 
                \rightarrow B\}}}} $\\
                $\displaystyle {{{}\over{\Gamma, x:X, y:Y \vdash 
                    \lambda z:Z.  ((x\, z)\, (y \, z)):Z \rightarrow A 
                    \,|_{\mathcal{X}_2
                        \cup\{A,B\}}\, 
                    \mathcal{C}_{2}\cup\{B\triangleq S_2\rightarrow A,
                    X \triangleq  Z 
                    \rightarrow B\}}}} $\\
                $\displaystyle {{{}\over{\Gamma, x:X \vdash \lambda 
                    y:Y.\lambda z:Z.  ((x\, z)\, (y \, z)):Y 
                    \rightarrow Z\rightarrow 
                    A\,|_{\mathcal{X}_2\cup\{A,B\}}\, 
                    \mathcal{C}_{2}\cup\{B\triangleq S_2\rightarrow A,
                    X \triangleq  Z 
                    \rightarrow B\}}}}  $\\
                $\displaystyle {{}
                    \over{\Gamma \vdash \lambda x:X. \lambda y:Y.\lambda z:Z.
                    ((x\, z)\, (y \, z)):X \rightarrow Y \rightarrow 
                Z\rightarrow A\,|_{\mathcal{X}_2
                    \cup\{A,B\}}\, 
                    \mathcal{C}_{2}\cup\{B\triangleq S_2\rightarrow A,
                    X \triangleq  Z 
                    \rightarrow B\}}} $
                \end{tabular}
            \onslide<9>
                \begin{tabular}{c}
                    \begin{tabular}{rl}
                        \begin{tabular}{c}
%                            \begin{tabular}{rl}
%                                \begin{tabular}{c}
%                                    $\displaystyle {{{}\over{x:X \in 
%                                    \Gamma, x:X}}}$\\
%                                    $\displaystyle {{{}\over{\Gamma, 
%                                    x:X \vdash x :X\,|_{\emptyset}\, 
%                                    \emptyset}}} $
%                                \end{tabular} &
%                                \begin{tabular}{c}
%                                    $\displaystyle {{{}\over{z:Z \in 
%                                    \Gamma, z:Z}}}$\\
%                                    $\displaystyle {{{}\over{\Gamma, 
%                                    z:Z \vdash z :Z\,|_{\emptyset}\, 
%                                    \emptyset}}} $
%                                \end{tabular}
%                            \end{tabular}\\
                            $\displaystyle {{{}\over{\Gamma, x:X, z:Z 
                            \vdash
                            x\, z:B\,|_{{\{B\}}}\, \{X \triangleq  Z \rightarrow 
                            B\}}}} $
                            \end{tabular}&
                        \begin{tabular}{c}
                            \begin{tabular}{rl}
                                \begin{tabular}{c}
                                    $\displaystyle {{{}\over{\Gamma, 
                                    y:Y \vdash y 
                                    :S_5\,|_{\mathcal{X}_5}\, 
                                    \mathcal{C}_5}}} $
                                \end{tabular} &
                                \begin{tabular}{c}
                                    $\displaystyle {{{}\over{\Gamma, 
                                    z:Z \vdash z 
                                    :S_6\,|_{\mathcal{X}_6}\, 
                                    \mathcal{C}_6}}} $
                                \end{tabular}
                            \end{tabular}\\
                            $\displaystyle {{{}\over{\Gamma, y:Y, z:Z 
                            \vdash
                            y \, 
                            z:C\,|_{\mathcal{X}_5\cup\mathcal{X}_6\cup
                            \{C\}}\, 
                            \mathcal{C}_5\cup\mathcal{C}_6\cup\{
                            S_5 \triangleq  S_6 \rightarrow C\}}}} $                      
                    \end{tabular}
                    \end{tabular}\\
                $\displaystyle {{{}\over{\Gamma, x:X, y:Y, z:Z \vdash
                (x\, z)\, (y \, 
                z):A\,|_{\mathcal{X}_5\cup\mathcal{X}_6
                    \cup\{A,B,C\}}\, 
                \mathcal{C}_5\cup\mathcal{C}_6\cup\{B\triangleq C\rightarrow A, 
                X \triangleq  Z \rightarrow B, S_5 \triangleq  S_6 \rightarrow C\}}}} $\\
                $\displaystyle {{{}\over{\Gamma, x:X, y:Y \vdash 
                    \lambda z:Z.  ((x\, z)\, (y \, z)):Z \rightarrow A 
                    \,|_{\mathcal{X}_5\cup\mathcal{X}_6
                        \cup\{A,B,C\}}\, 
                    \mathcal{C}_5\cup\mathcal{C}_6\cup\{B\triangleq C\rightarrow 
                    A, X \triangleq  Z \rightarrow B, S_5 \triangleq  S_6 \rightarrow 
                    C\}}}} $\\
                $\displaystyle {{{}\over{\Gamma, x:X \vdash \lambda 
                    y:Y.\lambda z:Z.  ((x\, z)\, (y \, z)):Y 
                    \rightarrow Z\rightarrow 
                    A\,|_{\mathcal{X}_5\cup\mathcal{X}_6\cup\{A,B,C\}}\, 
                    \mathcal{C}_5\cup\mathcal{C}_6\cup\{B\triangleq C\rightarrow 
                    A, X \triangleq  Z \rightarrow B, S_5 \triangleq  S_6 \rightarrow 
                    C\}}}}  $\\
                $\displaystyle {{}
                    \over{\Gamma \vdash \lambda x:X. \lambda y:Y.\lambda z:Z.
                    ((x\, z)\, (y \, z)):X \rightarrow Y \rightarrow 
                Z\rightarrow A\,|_{\mathcal{X}_5\cup\mathcal{X}_6
                    \cup\{A,B,C\}}\, 
                    \mathcal{C}_5\cup\mathcal{C}_6\cup\{B\triangleq C\rightarrow 
                    A, X \triangleq  Z \rightarrow B, S_5 \triangleq  S_6 \rightarrow 
                    C\}}} $
                \end{tabular}
            \onslide<10>
                \begin{tabular}{c}
                    \begin{tabular}{rl}
                        \begin{tabular}{c}
%                            \begin{tabular}{rl}
%                                \begin{tabular}{c}
%                                    $\displaystyle {{{}\over{x:X \in 
%                                    \Gamma, x:X}}}$\\
%                                    $\displaystyle {{{}\over{\Gamma, 
%                                    x:X \vdash x :X\,|_{\emptyset}\, 
%                                    \emptyset}}} $
%                                \end{tabular} &
%                                \begin{tabular}{c}
%                                    $\displaystyle {{{}\over{z:Z \in 
%                                    \Gamma, z:Z}}}$\\
%                                    $\displaystyle {{{}\over{\Gamma, 
%                                    z:Z \vdash z :Z\,|_{\emptyset}\, 
%                                    \emptyset}}} $
%                                \end{tabular}
%                            \end{tabular}\\
                            $\displaystyle {{{}\over{\Gamma, x:X, z:Z 
                            \vdash
                            x\, z:B\,|_{{\{B\}}}\, \{X \triangleq  Z \rightarrow 
                            B\}}}} $
                            \end{tabular}&
                        \begin{tabular}{c}
                            \begin{tabular}{rl}
                                \begin{tabular}{c}
                                    $\displaystyle {{{}\over{y:Y \in 
                                    \Gamma, y:Y}}}$\\
                                    $\displaystyle {{{}\over{\Gamma, 
                                    y:Y \vdash y :Y\,|_{\emptyset}\, 
                                    \emptyset}}} $
                                \end{tabular} &
                                \begin{tabular}{c}
                                    $\displaystyle {{{}\over{\Gamma, 
                                    z:Z \vdash z 
                                    :S_6\,|_{\mathcal{X}_6}\, 
                                    \mathcal{C}_6}}} $
                                \end{tabular}
                            \end{tabular}\\
                            $\displaystyle {{{}\over{\Gamma, y:Y, z:Z 
                            \vdash
                            y \, z:C\,|_{\mathcal{X}_6\cup
                            \{C\}}\, \mathcal{C}_6\cup\{
                            Y \triangleq  S_6 \rightarrow C\}}}} $                      
                    \end{tabular}
                    \end{tabular}\\
                $\displaystyle {{{}\over{\Gamma, x:X, y:Y, z:Z \vdash
                (x\, z)\, (y \, z):A\,|_{\mathcal{X}_6
                    \cup\{A,B,C\}}\, \mathcal{C}_6\cup\{B\triangleq C\rightarrow 
                A, X \triangleq  Z \rightarrow B, Y \triangleq  S_6 \rightarrow C\}}}} $\\
                $\displaystyle {{{}\over{\Gamma, x:X, y:Y \vdash 
                    \lambda z:Z.  ((x\, z)\, (y \, z)):Z \rightarrow A 
                    \,|_{\mathcal{X}_6
                        \cup\{A,B,C\}}\, 
                    \mathcal{C}_6\cup\{B\triangleq C\rightarrow A, X
                    \triangleq  Z 
                    \rightarrow B, Y \triangleq  S_6 \rightarrow C\}}}} $\\
                $\displaystyle {{{}\over{\Gamma, x:X \vdash \lambda 
                    y:Y.\lambda z:Z.  ((x\, z)\, (y \, z)):Y 
                    \rightarrow Z\rightarrow 
                    A\,|_{\mathcal{X}_6\cup\{A,B,C\}}\, 
                    \mathcal{C}_6\cup\{B\triangleq C\rightarrow A, X
                    \triangleq  Z 
                    \rightarrow B, Y \triangleq  S_6 \rightarrow C\}}}}  $\\
                $\displaystyle {{}
                    \over{\Gamma \vdash \lambda x:X. \lambda y:Y.\lambda z:Z.
                    ((x\, z)\, (y \, z)):X \rightarrow Y \rightarrow 
                Z\rightarrow A\,|_{\mathcal{X}_6
                    \cup\{A,B,C\}}\, \mathcal{C}_6\cup\{B\triangleq C\rightarrow 
                    A, X \triangleq  Z \rightarrow B, Y \triangleq  S_6 \rightarrow C\}}} 
                    $
                \end{tabular}
            \onslide<11>
                \begin{tabular}{c}
                    \begin{tabular}{rl}
                        \begin{tabular}{c}
%                            \begin{tabular}{rl}
%                                \begin{tabular}{c}
%                                    $\displaystyle {{{}\over{x:X \in 
%                                    \Gamma, x:X}}}$\\
%                                    $\displaystyle {{{}\over{\Gamma, 
%                                    x:X \vdash x :X\,|_{\emptyset}\, 
%                                    \emptyset}}} $
%                                \end{tabular} &
%                                \begin{tabular}{c}
%                                    $\displaystyle {{{}\over{z:Z \in 
%                                    \Gamma, z:Z}}}$\\
%                                    $\displaystyle {{{}\over{\Gamma, 
%                                    z:Z \vdash z :Z\,|_{\emptyset}\, 
%                                    \emptyset}}} $
%                                \end{tabular}
%                            \end{tabular}\\
                            $\displaystyle {{{}\over{\Gamma, x:X, z:Z 
                            \vdash
                            x\, z:B\,|_{{\{B\}}}\, \{X \triangleq  Z \rightarrow 
                            B\}}}} $
                            \end{tabular}&
                        \begin{tabular}{c}
                            \begin{tabular}{rl}
                                \begin{tabular}{c}
                                    $\displaystyle {{{}\over{y:Y \in 
                                    \Gamma, y:Y}}}$\\
                                    $\displaystyle {{{}\over{\Gamma, 
                                    y:Y \vdash y :Y\,|_{\emptyset}\, 
                                    \emptyset}}} $
                                \end{tabular} &
                                \begin{tabular}{c}
                                    $\displaystyle {{{}\over{z:Z \in 
                                    \Gamma, z:Z}}}$\\
                                    $\displaystyle {{{}\over{\Gamma, 
                                    z:Z \vdash z :Z\,|_{\emptyset}\, 
                                    \emptyset}}} $
                                \end{tabular}
                            \end{tabular}\\
                            $\displaystyle {{{}\over{\Gamma, y:Y, z:Z 
                            \vdash
                            y \, z:C\,|_{
                            \{C\}}\, \{
                            Y \triangleq  Z \rightarrow C\}}}} $                      
                    \end{tabular}
                    \end{tabular}\\
                $\displaystyle {{{}\over{\Gamma, x:X, y:Y, z:Z \vdash
                (x\, z)\, (y \, z):A\,|_{\{A,B,C\}}\, \{B\triangleq C\rightarrow 
                A, X \triangleq  Z \rightarrow B, Y \triangleq  Z \rightarrow C\}}}} $\\
                $\displaystyle {{{}\over{\Gamma, x:X, y:Y \vdash 
                    \lambda z:Z.  ((x\, z)\, (y \, z)):Z \rightarrow A 
                    \,|_{\{A,B,C\}}\, \{B\triangleq C\rightarrow A, X
                    \triangleq Z 
                    \rightarrow B, Y \triangleq  Z \rightarrow C\}}}} $\\
                $\displaystyle {{{}\over{\Gamma, x:X \vdash \lambda 
                    y:Y.\lambda z:Z.  ((x\, z)\, (y \, z)):Y 
                    \rightarrow Z\rightarrow A\,|_{\{A,B,C\}}\, 
                    \{B\triangleq C\rightarrow A, X \triangleq  Z
                    \rightarrow B, Y \triangleq  Z 
                    \rightarrow C\}}}}  $\\
                $\displaystyle {{}
                    \over{\Gamma \vdash \lambda x:X. \lambda y:Y.\lambda z:Z.
                    ((x\, z)\, (y \, z)):X \rightarrow Y \rightarrow 
                Z\rightarrow A\,|_{\{A,B,C\}}\, \{B\triangleq C\rightarrow A, X 
                \triangleq  Z \rightarrow B, Y \triangleq  Z \rightarrow C\}}} $
                \end{tabular}
            \onslide<12>
                \begin{tabular}{c}
                    \begin{tabular}{rl}
                        \begin{tabular}{c}
%                            \begin{tabular}{rl}
%                                \begin{tabular}{c}
%                                    $\displaystyle {{{}\over{x:X \in 
%                                    \Gamma, x:X}}}$\\
%                                    $\displaystyle {{{}\over{\Gamma, 
%                                    x:X \vdash x :X\,|_{\emptyset}\, 
%                                    \emptyset}}} $
%                                \end{tabular} &
%                                \begin{tabular}{c}
%                                    $\displaystyle {{{}\over{z:Z \in 
%                                    \Gamma, z:Z}}}$\\
%                                    $\displaystyle {{{}\over{\Gamma, 
%                                    z:Z \vdash z :Z\,|_{\emptyset}\, 
%                                    \emptyset}}} $
%                                \end{tabular}
%                            \end{tabular}\\
                            $\displaystyle {{{}\over{\Gamma, x:X, z:Z 
                            \vdash
                            x\, z:B\,|_{{\{B\}}}\, \{X \triangleq  Z \rightarrow 
                            B\}}}} $
                            \end{tabular}&
                        \begin{tabular}{c}
%                            \begin{tabular}{rl}
%                                \begin{tabular}{c}
%                                    $\displaystyle {{{}\over{y:Y \in 
%                                    \Gamma, y:Y}}}$\\
%                                    $\displaystyle {{{}\over{\Gamma, 
%                                    y:Y \vdash y :Y\,|_{\emptyset}\, 
%                                    \emptyset}}} $
%                                \end{tabular} &
%                                \begin{tabular}{c}
%                                    $\displaystyle {{{}\over{z:Z \in 
%                                    \Gamma, z:Z}}}$\\
%                                    $\displaystyle {{{}\over{\Gamma, 
%                                    z:Z \vdash z :Z\,|_{\emptyset}\, 
%                                    \emptyset}}} $
%                                \end{tabular}
%                            \end{tabular}\\
                            $\displaystyle {{{}\over{\Gamma, y:Y, z:Z 
                            \vdash
                            y \, z:C\,|_{
                            \{C\}}\, \{
                            Y \triangleq  Z \rightarrow C\}}}} $                      
                    \end{tabular}
                    \end{tabular}\\
                $\displaystyle {{{}\over{\Gamma, x:X, y:Y, z:Z \vdash
                (x\, z)\, (y \, z):A\,|_{\{A,B,C\}}\, \{B\triangleq C\rightarrow 
                A, X \triangleq  Z \rightarrow B, Y \triangleq  Z \rightarrow C\}}}} $\\
                $\displaystyle {{{}\over{\Gamma, x:X, y:Y \vdash 
                    \lambda z:Z.  ((x\, z)\, (y \, z)):Z \rightarrow A 
                    \,|_{\{A,B,C\}}\, \{B\triangleq C\rightarrow A, X
                    \triangleq Z 
                    \rightarrow B, Y \triangleq  Z \rightarrow C\}}}} $\\
                $\displaystyle {{{}\over{\Gamma, x:X \vdash \lambda 
                    y:Y.\lambda z:Z.  ((x\, z)\, (y \, z)):Y 
                    \rightarrow Z\rightarrow A\,|_{\{A,B,C\}}\, 
                    \{B\triangleq C\rightarrow A, X \triangleq  Z
                    \rightarrow B, Y \triangleq  Z 
                    \rightarrow C\}}}}  $\\
                $\displaystyle {{}
                    \over{\Gamma \vdash \lambda x:X. \lambda y:Y.\lambda z:Z.
                    ((x\, z)\, (y \, z)):X \rightarrow Y \rightarrow 
                Z\rightarrow A\,|_{\{A,B,C\}}\, \{B\triangleq C\rightarrow A, X 
                \triangleq  Z \rightarrow B, Y \triangleq  Z \rightarrow C\}}} $
                \end{tabular}
                \begin{block}{Conclusion}
                    We conclude that the term $\lambda x:X. \lambda y:Y.
                    \lambda z:Z.  ((x\, z)\, (y \, z))$ has
                    \emph{abstract type} $X
                    \rightarrow Y \rightarrow Z\rightarrow A$ if it is
                    possible to find a type substitution $\sigma$ such
                    that $\sigma \Join \{B\triangleq C\rightarrow A, X 
                    \triangleq  Z \rightarrow B, Y \triangleq  Z \rightarrow C\}$
                \end{block}
    \end{overprint}
    \end{center}
    \normalsize
\end{frame}

\begin{frame}{Extended example}
    \footnotesize
        \begin{overprint}
            \onslide<1>
            \begin{tabular}{c}
                $\displaystyle {{}\over{\Gamma \vdash \lambda x:X.  
                (x\, x):S_1\,|_{\mathcal{X}}\, \mathcal{C}}}$
            \end{tabular}
            \onslide<2>
                \begin{tabular}{c}
                    $\displaystyle {{{}\over{\Gamma, x:X \vdash
                        x\, x:S_2\,|_{\mathcal{X}}\, \mathcal{C}}}} $\\
                    $\displaystyle {{}\over{\Gamma \vdash \lambda x:X.  
                    (x\, x):X\rightarrow S_2\,|_{\mathcal{X}}\, \mathcal{C}}}$
                \end{tabular}
            \onslide<3>
                \begin{tabular}{c}
                    \begin{tabular}{cc}
                        \begin{tabular}{c}
                            $\displaystyle {{{}\over{\Gamma, x:X 
                            \vdash
                            x:S_1\,|_{\mathcal{X}_1}\, 
                            \mathcal{C}_1}}} $                        
                            \end{tabular}&
                        \begin{tabular}{c}
                            $\displaystyle {{{}\over{\Gamma, x:X 
                            \vdash
                            x:S_2\,|_{\mathcal{X}_2}\, 
                            \mathcal{C}_2}}} $                        
                            \end{tabular}
                    \end{tabular}\\
                    $\displaystyle {{{}\over{\Gamma, x:X \vdash
                        x\, x:A\,|_{\mathcal{X}_1\cup\mathcal{X}_2\cup
                        \{A\}}\, \mathcal{C}_1\cup\mathcal{C}_2\cup
                        \{S_1 \triangleq S_2\rightarrow A\}}}} $\\
                    $\displaystyle {{}\over{\Gamma \vdash \lambda x:X.  
                    (x\, x):X\rightarrow A\,|_{\mathcal{X}_1
                    \cup\mathcal{X}_2\cup \{A\}}\, \mathcal{C}_1\cup
                        \mathcal{C}_2\cup \{S_1 \triangleq S_2\rightarrow A\}}}$
                \end{tabular}
            \onslide<4>
                \begin{tabular}{c}
                    \begin{tabular}{cc}
                        \begin{tabular}{c}
                            $\displaystyle {{{}\over{x:X\in \Gamma,
                            x:X }}} $  \\
                            $\displaystyle {{{}\over{\Gamma, x:X 
                            \vdash
                            x:X\,|_{\emptyset}\, 
                            \emptyset}}} $                        
                            \end{tabular}&
                        \begin{tabular}{c}
                            $\displaystyle {{{}\over{\Gamma, x:X 
                            \vdash
                            x:S_2\,|_{\mathcal{X}_2}\, 
                            \mathcal{C}_2}}} $                        
                            \end{tabular}
                    \end{tabular}\\
                    $\displaystyle {{{}\over{\Gamma, x:X \vdash 
                        x\, x:A\,|_{\mathcal{X}_2\cup
                        \{A\}}\, \mathcal{C}_2\cup
                        \{X \triangleq S_2\rightarrow A\}}}} $\\
                    $\displaystyle {{}\over{\Gamma \vdash \lambda x:X.  
                    (x\, x):X\rightarrow A\,|_{
                        \mathcal{X}_2\cup \{A\}}\, 
                        \mathcal{C}_2\cup \{X \triangleq S_2\rightarrow A\}}}$
                \end{tabular}
            \onslide<5>
                \begin{tabular}{c}
                    \begin{tabular}{cc}
                        \begin{tabular}{c}
                            $\displaystyle {{{}\over{x:X\in \Gamma,
                            x:X }}} $  \\
                            $\displaystyle {{{}\over{\Gamma, x:X 
                            \vdash
                            x:X\,|_{\emptyset}\, 
                            \emptyset}}} $                        
                            \end{tabular}&
                        \begin{tabular}{c}
                            $\displaystyle {{{}\over{x:X\in \Gamma,
                            x:X }}} $  \\
                            $\displaystyle {{{}\over{\Gamma, x:X 
                            \vdash
                            x:X\,|_{\emptyset}\, 
                            \emptyset}}} $                        
                            \end{tabular}
                    \end{tabular}\\
                    $\displaystyle {{{}\over{\Gamma, x:X \vdash
                        x\, x:A\,|_{
                        \{A\}}\, \{X \triangleq X\rightarrow A\}}}} $\\
                    $\displaystyle {{}\over{\Gamma \vdash \lambda x:X.  
                    (x\, x):X\rightarrow A\,|_{
                        \{A\}}\,  \{X \triangleq X\rightarrow A\}}}$
                \end{tabular}
                \begin{block}{Conclusion}
                    We conclude that the term $\lambda x:X.(x \, x))$ has
                    \emph{abstract type} $X\rightarrow A$ if it is
                    possible to find a type substitution $\sigma$ such
                    that $\sigma \Join \{X\triangleq X\rightarrow A\}$
                \end{block}
        \end{overprint}
    \normalsize
\end{frame}

\subsection{Definition of solution for $(\Gamma, t, S, \mathcal{C})$}

\begin{frame}
    It is helpful make the following observations:
    \begin{itemize}
        \item when a type variable $X$ is choosen by a final rule which has
            some premises, then $X$ is different from all other type variables
            introduced by premises' subderivations
        \item for any given pair $(\Gamma, t)$ the rules provide a procedure to 
            build sets $\mathcal{X}, \mathcal{C}$ and find type $T_1$ such that
            $\Gamma \vdash t : T_1 \; \textbar_\mathcal{X} \; \mathcal{C}$
        \item if we consider the relation  
            $\Gamma \vdash t : T_1 \; \textbar_\mathcal{X} \; \mathcal{C}$ modulo
            the choice of fresh variables, then the constraint set $\mathcal{C}$ and
            type $T_1$ are \emph{uniquely} determined
        \item in order to find solutions for $(\Gamma, t)$ we use the given rules to:
            \begin{enumerate}
                \item build the constraint set $\mathcal{C}$, that
                    must be satisfied in order for $t$ to have a type
                \item determine a type $S$ possibly containing type variables (which are 
                    subjects under constraints in $\mathcal{C}$), which caracterizes
                    the types of $t$ in terms of these variables
                \item find a type sustitution $\sigma$ such that $\sigma \Join 
                    \mathcal{C}$: for each such $\sigma$ the type $\sigma S$ is
                    a possible type for $t$, hence $(\sigma, \sigma S)$ 
                    is a solution for $(\Gamma, t)$
                \item if no type sustitution $\sigma \Join \mathcal{C}$ exists then
                    there is no way to instantiate type variables in $t$ in order
                    for $t$ to have a type.
            \end{enumerate}
    \end{itemize}
\end{frame}


\begin{frame}
    Let $\Gamma$ be a context and $t$ be a term containing free variables.
    \begin{block}{Definition of \emph{solution} for $(\Gamma, t, S, \mathcal{C})$}
        Let $\sigma$ be a type substitution and $T_{1} \in T$ and suppose
            $\Gamma \vdash t : T_1 \; \textbar_\mathcal{X} \; \mathcal{C}$ \\
        A  \emph{solution} for $(\Gamma, t, S, \mathcal{C})$ is a pair 
        $(\sigma, T_{1})$ such that $\sigma \Join \mathcal{C} \wedge \sigma S = T_1$
    \end{block}
    
    \begin{example}
        Let $t = \lambda x:X\rightarrow Y. x \, 0$ then:
        \begin{displaymath}
            \begin{split}
                S &= (X \rightarrow Y) \rightarrow Z \\
                \mathcal{C} &= \{Nat \rightarrow Z \triangleq X \rightarrow Y\}\\
                \sigma &= \{X \mapsto Nat, Z \mapsto Bool, Y \mapsto Bool\}\\
            \end{split}
        \end{displaymath}
        hence $(\sigma, (Nat \rightarrow Bool) \rightarrow Bool)$ is a solution 
        for $(\Gamma, t, S, \mathcal{C})$.
    \end{example}
\end{frame}

\begin{frame}
    Let $\Gamma$ be a context and $t$ be a term containing free variables\\~\\
    We have two different ways of instantiating type variables appearing in $t$
    to produce a typeable term. In the next definitions $\sigma$ is a type
    sustitution and $T_1 \in T$ as usual:
    \begin{block}{Declarative approach}
        $\Omega = \{ \omega = (\sigma, T_1): \omega\text{ is solution of } (\Gamma, t)\}$
    \end{block}

    \begin{block}{Algorithmic approach}
        $\Theta = \{ \theta = (\sigma, T_1): \theta\text{ is solution of } 
            (\Gamma, t, S, \mathcal{C})\}$
    \end{block}
    
    

    \begin{theorem}
    $\Omega = \Theta$    
    \end{theorem}
\end{frame}


\section{Unification}
\subsection{Algorithm}
\begin{frame}
    \begin{block}{Unification algorithm}
        By pattern matching on the structure of constraint set $\mathcal{C}$
        given as argument:
        \begin{displaymath}
            \begin{split}
                unify(\emptyset) &= [\,] \\
                unify(\{T_1 \triangleq T_2\} \cup \mathcal{C}^{\prime}) &=
                    unify(\mathcal{C}^{\prime})
                    \quad \text{if } T_1 = T_2\\
                unify(\{X \triangleq T_1\} \cup \mathcal{C}^{\prime}) &=
                    unify([ X \mapsto T_1]\mathcal{C}^{\prime}) \circ [X \mapsto T_1]
                    \quad \text{if } X \not \in FV(T_1)\\
                unify(\{T_1 \triangleq X\} \cup \mathcal{C}^{\prime}) &=
                    unify([ X \mapsto T_1]\mathcal{C}^{\prime}) \circ [X \mapsto T_1]
                    \quad \text{if } X \not \in FV(T_1)\\
                unify(\{T_1 \rightarrow T_2 \triangleq S_1\rightarrow S_2 \} 
                    \cup \mathcal{C}^{\prime}) &=
                    unify(\mathcal{C}^{\prime} \cup \{T_1 \triangleq S_1,
                        T_2 \triangleq S_2\}) \\
                unify(\_) &= \text{\color{red} raise failure}\\
            \end{split}
        \end{displaymath}
    \end{block}
    
    For the properties' discussion are useful the following concepts:
    \begin{itemize}
        \item   A type substitution $\sigma$ is \emph{less specific} (or 
                \emph{more general}) than a type substitution $\rho$, written
                as $\sigma \sqsubseteq \rho$, if $\rho = \gamma \circ \sigma$,
                for some type substitution $\gamma$
        \item   A \emph{principal unifier} for a constraint set $\mathcal{C}$ 
                is a type substitution $\sigma$ such that $\sigma \Join \mathcal{C}$
                and $\forall \rho \Join \mathcal{C}: \sigma \sqsubseteq \rho$
    \end{itemize}
    \begin{example}
        Let $\rho = \{S \mapsto Nat \rightarrow Nat, Z \mapsto Nat\}$ 
        and $\sigma = \{S \mapsto Z \rightarrow Z\}$. We have $\sigma 
        \sqsubseteq \rho$ because $\exists \gamma = \{Z \mapsto Nat\}$ 
        such that $\rho = \gamma \circ \sigma$
    \end{example}
\end{frame}

\begin{frame}
    We return to the extended example: we built the constraint set
    $\mathcal{C} = \{B\triangleq C\rightarrow A, X 
                        \triangleq  Z \rightarrow B, Y \triangleq  Z
                        \rightarrow C\}$, now we apply the unification
                        in order to build a type substitution $\sigma$
                        (hoping such that $\sigma \Join \mathcal{C}$)
    \begin{overprint}
        \onslide<1>
    \begin{displaymath}
             unify(\{B\triangleq C\rightarrow A,X 
                                \triangleq  Z \rightarrow B, Y
                                \triangleq  Z \rightarrow C\}) \\
    \end{displaymath}
        \onslide<2>
    \begin{displaymath}
             unify(\{B\triangleq C\rightarrow A\} \cup \{X 
                                \triangleq  Z \rightarrow B, Y
                                \triangleq  Z \rightarrow C\}) \\
    \end{displaymath}
        \onslide<3>
    \begin{displaymath}
             unify([B\mapsto C\rightarrow A]\{X \triangleq  Z \rightarrow B, Y
                                \triangleq  Z \rightarrow
                                C\})\circ[B\mapsto C\rightarrow A]\\
    \end{displaymath}
        \onslide<4>
    \begin{displaymath}
             unify(\{ X \triangleq  Z \rightarrow C\rightarrow A, Y
                                \triangleq  Z \rightarrow
                                C\})\circ[B\mapsto C\rightarrow A]\\
    \end{displaymath}
        \onslide<5>
    \begin{displaymath}
             unify(\{ X \triangleq  Z \rightarrow C\rightarrow
                A\}\cup\{ Y \triangleq  Z \rightarrow
                                C\})\circ[B\mapsto C\rightarrow A]\\
    \end{displaymath}
        \onslide<6>
    \begin{displaymath}
             unify([ X \triangleq  Z \rightarrow C\rightarrow
                A]\{ Y \triangleq  Z \rightarrow
                    C\})\circ[ X \triangleq  Z \rightarrow C\rightarrow
                A]\circ[B\mapsto C\rightarrow A]\\
    \end{displaymath}
        \onslide<7>
    \begin{displaymath}
             unify(\{ Y \triangleq  Z \rightarrow
                    C\})\circ[ X \triangleq  Z \rightarrow C\rightarrow
                A]\circ[B\mapsto C\rightarrow A]\\
    \end{displaymath}
        \onslide<8>
    \begin{displaymath}
             unify(\emptyset)\circ[ Y \triangleq  Z \rightarrow
                    C]\circ[ X \triangleq  Z \rightarrow C\rightarrow
                A]\circ[B\mapsto C\rightarrow A]\\
    \end{displaymath}
        \onslide<9>
    \begin{displaymath}
             []\circ[ Y \triangleq  Z \rightarrow
                    C]\circ[ X \triangleq  Z \rightarrow C\rightarrow
                A]\circ[B\mapsto C\rightarrow A]\\
    \end{displaymath}
    \end{overprint}
\end{frame}

\subsection{Properties}

\begin{frame}
    \begin{theorem}
        $unify(\mathcal{C})$ halts, either by failing or by returning a type 
        substitution $\sigma$
    \end{theorem}
    \begin{theorem}
        $unify(\mathcal{C}) = \sigma \rightarrow \sigma \Join \mathcal{C}$
    \end{theorem}
    \begin{theorem}
        $\rho \Join \mathcal{C} \rightarrow unify(\mathcal{C}) = \sigma \wedge 
            \sigma \sqsubseteq \rho$
    \end{theorem}
\end{frame}

\subsection{Definition of principal solution for $(\Gamma, t,S, \mathcal{C})$}

\begin{frame}
    Let $\Gamma$ be a context and $t$ be a term containing free variables.
    \begin{block}{Definition of \emph{principal solution} for 
            $(\Gamma, t, S, \mathcal{C})$}
        A  \emph{principal solution} for $(\Gamma, t, S, \mathcal{C})$ 
        is a solution $(\sigma, T_{1})$ such that for any other solution
        $(\rho, T_2)$ for $(\Gamma, t, S, \mathcal{C})$ we have $\sigma 
        \sqsubseteq \rho$.
    \end{block}
    
    \begin{theorem}
        if $(\Gamma, t, S, \mathcal{C})$ has a solution, then it has a 
        principal one too. The unification algorithm can be used to
        decide if $(\Gamma, t, S, \mathcal{C})$ has solutions and if it 
        is the case, it compute the principal one.
    \end{theorem}
\end{frame}

\begin{frame}
    We finish our extended example to show a principal solution. Using
    unification algorithm we found the type substitution:
    \begin{displaymath}
         \sigma = []\circ[ Y \triangleq  Z \rightarrow
                C]\circ[ X \triangleq  Z \rightarrow C\rightarrow
            A]\circ[B\mapsto C\rightarrow A]
    \end{displaymath}
    And, using the deduction for constraint typing relation, we found
    that:
    \begin{displaymath}
        \lambda x:X. \lambda y:Y.
            \lambda z:Z.  ((x\, z)\, (y \, z)):X \rightarrow
                Y \rightarrow Z\rightarrow A
    \end{displaymath}
    In order to find a principal solution we apply $\sigma$ to the
    abstract type above to have a \emph{concrete} type for the term
    (possibly containing free variables, heart of parametric polymorphism,
    remember?):
    \begin{overprint}
        \onslide<1>
        \begin{displaymath}
            \sigma (X \rightarrow Y \rightarrow Z\rightarrow A)
        \end{displaymath}
        \onslide<2>
        \begin{displaymath}
            \sigma X \rightarrow \sigma(Y \rightarrow Z\rightarrow A)
        \end{displaymath}
        \onslide<3>
        \begin{displaymath}
            \sigma X \rightarrow \sigma Y \rightarrow \sigma (Z\rightarrow A)
        \end{displaymath}
        \onslide<4>
        \begin{displaymath}
            \sigma X \rightarrow \sigma Y \rightarrow \sigma
            Z\rightarrow \sigma A
        \end{displaymath}
        \onslide<5>
        \begin{displaymath}
            (Z\rightarrow C\rightarrow A) \rightarrow \sigma Y 
            \rightarrow \sigma Z\rightarrow \sigma A
        \end{displaymath}
        \onslide<6>
        \begin{displaymath}
            (Z\rightarrow C\rightarrow A) \rightarrow (Z\rightarrow C) 
            \rightarrow \sigma Z\rightarrow \sigma A
        \end{displaymath}
        \onslide<7>
        \begin{displaymath}
            (Z\rightarrow C\rightarrow A) \rightarrow (Z\rightarrow C) 
            \rightarrow Z\rightarrow \sigma A
        \end{displaymath}
        \onslide<8>
        \begin{displaymath}
            (Z\rightarrow C\rightarrow A) \rightarrow (Z\rightarrow C) 
            \rightarrow Z\rightarrow  A
        \end{displaymath}
        Hence $(\sigma,             (Z\rightarrow C\rightarrow A)
        \rightarrow (Z\rightarrow C) \rightarrow Z\rightarrow  A),
        \forall A,C,Z\in T,$ is the principal solution for 
        \begin{displaymath}    
                (\emptyset, \lambda x:X. \lambda y:Y.
            \lambda z:Z.  ((x\, z)\, (y \, z)), X \rightarrow
                Y \rightarrow Z\rightarrow A, \{B\triangleq
                C\rightarrow A, X 
                                        \triangleq  Z \rightarrow B, Y
                                        \triangleq  Z
                                                                \rightarrow
                                                                C\})
        \end{displaymath}
    \end{overprint}
\end{frame}


\end{document}

